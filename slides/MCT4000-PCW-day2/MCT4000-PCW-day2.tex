\documentclass[screen, aspectratio=169]{beamer}
\usepackage[T1]{fontenc}
\usepackage[utf8]{inputenc}

% Use the NTNU-temaet for beamer 
% \usetheme[style=ntnu|simple|vertical|horizontal, 
%     language=bm|nn|en, 
%     smalltitle, 
%     city=all|trondheim|alesund|gjovik]{ntnu2017}
\usetheme[style=ntnu, language=en, smalltitle]{ntnu2017}

\usepackage[english]{babel}
\usepackage[style=numeric,backend=biber,natbib=false,sorting=none]{biblatex}

\title[PCW-d1]{Physical Computing Workshop: Day 2}
\subtitle{Embedded Hacking}
\author[A. Xamb{\'o}]{Anna Xamb{\'o}}
\institute[NTNU]{Department of Music, NTNU}
\date{16 October 2019}
%\date{} % To have an empty date

\addbibresource{../pcw.bib} % Add bibliography database

% Set the reference style to numeric.
% See here: http://tex.stackexchange.com/questions/68080/beamer-bibliography-icon
\setbeamertemplate{bibliography item}[text] 

% Set bibliography fonts to a small size.
\renewcommand*{\bibfont}{\footnotesize}

\begin{document}

\begin{frame}
  \titlepage
\end{frame}
%
%\begin{frame}
%\frametitle{Setting up}
%\begin{itemize}
%\item 
%\end{itemize}
%\end{frame}
%
\usebackgroundtemplate{}
\begin{frame}
\frametitle{}
{\huge Warm-up Activity\\Yesterday's blogpost and performance }
\end{frame}
%
\begin{frame}
\frametitle{Learning Outcomes}
By the end of the session,  you will be able to...
\begin{itemize}
\item Get familiar with Puredata and Bela and how to work with the two.
\item Get a sense of how what is an IDE and how the Bela IDE works.
\item Be able to prototype simple ``patches'' using Puredata, Bela and the breadboard.
\item Explore mappings from sensor data to sound. 
\item Explore adapting some concepts from the ``Victorian synthesizer'' to the digital domain.
\item Demonstrate a custom-made musical instrument in a performance setting.
\end{itemize}
\end{frame}
%

%    \item Identify the basic differences between sensors and actuators.
%    \item Get a sense of how sensor data works in mobile phones.
%    \item Be able to normalize sensor data.
%    \item Explore mappings from sensor data to sound.
%    \item Discern the fundamental properties of gestural interaction.   
%    \item Get familiar with web technologies (Handwaving.js) and web audio technologies (Tone.js, Flocking.js).
%    \item Be able to adapt javascript code of a gesture-driven musical piece with custom sounds.     
%    \item Demonstrate a custom-made musical instrument in a performance setting.
%    \item Reflect on the custom-made musical instrument and performance using a blogging style.   
%  \end{itemize}

\begin{frame}
  \frametitle{Preparation: Read/skim through the following readings}
  \begin{itemize}
  \item Pure Data
    \begin{itemize}
    \item Puckette, M. (1997): Pure Data: Another Integrated Computer Music Environment
    \item Chapter 9: Starting with Pure Data. Farnell, Andy. Designing sound. MIT Press, 2010.
    \item Chapter 10: Using Pure Data. Farnell, Andy. Designing sound. MIT Press, 2010.
    \item Chapter 11: Pure Data Audio. Farnell, Andy. Designing sound. MIT Press, 2010.
     \end{itemize}
     \item Bela
     \begin{itemize}
     \item McPherson, Andrew, and Victor Zappi. "An environment for submillisecond-latency audio and sensor processing on BeagleBone Black." Audio Engineering Society Convention 138. Audio Engineering Society, 2015.
     \end{itemize}
  \end{itemize}
\end{frame}
%
\begin{frame}
  \frametitle{Puredata: Template patterns }
       \begin{itemize}
     \item See folder ``Puredata'' in code day 2: Oscillators, Time, Random.
     \end{itemize}
\end{frame}
%
\begin{frame}
  \frametitle{Breadboard}
       \begin{figure}
	\includegraphics[scale=0.5]{img/breadboard-pinout.png}
	\end{figure}
{\tiny
Source: \url{https://components101.com/misc/breadboard-connections-uses-guide}
}
\end{frame}
%
\usebackgroundtemplate{}
\begin{frame}
\frametitle{}
{\huge Block I\\BELA: Overview }
\end{frame}
%
\begin{frame}
  \frametitle{What is Bela?}
\begin{itemize}
\item An environment for ultra-low-latency processing of audio and sensor data on embedded hardware.
\item Based on the low-cost BeagleBone Black single-board computer.
\item A custom expansion board features stereo audio and 8 channels each of 16-bit ADC and 16-bit DAC for sensors and actuators.
\item It achieves latency as low as 80 microseconds (0.08 milliseconds).
\item It combines the best aspects of embedded Linux systems and dedicated microcontrollers for real-time audio.
\end{itemize}
{\tiny
McPherson, Andrew, and Victor Zappi. "An environment for submillisecond-latency audio and sensor processing on BeagleBone Black." Audio Engineering Society Convention 138. Audio Engineering Society, 2015. \cite{Mcpherson.Zappi.2015.bela}
}
\end{frame}
%
\begin{frame}
  \frametitle{Bela vs\ Arduino}
\begin{itemize}
\item In general, microcontroller platforms offer easy connections to hardware sensors and predictable timing, but have limited computing power.
\item Embedded computers benefit from the ability to use familiar software tools (Pd, SuperCollider, ChucK, Python) and from the resources of a general-purpose OS, including file I/O and networking. They are optimised to balance many simultaneous processes but no guarantee to prioritize audio performance.
\item Arduino and similar microcontrollers are often connected by a serial port to connect to mobile devices / embedded computers. This limits sensor bandwidth causing low bit resolution of sensor data.
\end{itemize}
\end{frame}
%
\begin{frame}
  \frametitle{BeagleBone Board}
         \begin{figure}
	\includegraphics[scale=0.2]{img/beagleboneboard.png}
	\end{figure}
  \begin{itemize}
\item It is a single-board computer: 1GHz ARM Cortex-A8 processor, 512MB of RAM, 4GB of onboard storage.
\end{itemize}
\end{frame}
%
\begin{frame}
  \frametitle{Bela}
 \begin{itemize}
\item It is a custom hardware expansion board (``a cape'')  which provides stereo audio input and output, plus 8 channels each of 16-bit ADC and 16-bit DAC for sensors and actuators.
\item The board also contains onboard stereo 1.1W speaker amplifiers for making self-contained instruments.
\end{itemize}
\end{frame}
%
\begin{frame}
  \frametitle{Bela Hardware}
     \begin{figure}
	\includegraphics[scale=0.5]{img/bela-hardware.png}
\end{figure}
\end{frame}
%
\begin{frame}
  \frametitle{Bela Documentation}
 \begin{itemize}
\item Bela wiki on GitHub: \url{https://github.com/BelaPlatform/Bela/wiki}
\item Example projects and tutorials: \url{https://github.com/BelaPlatform/Bela/wiki/Example-projects-and-tutorials}
\end{itemize}
\end{frame}
%
%\begin{frame}
%  \frametitle{Bela: How to Start?}
%  GitHub official webpage:
% \begin{itemize}
%\item Getting started with Bela: \url{https://github.com/BelaPlatform/Bela/wiki/Getting-started-with-Bela#build-your-first-project}
%\item Building Bela projects with the IDE: \url{https://github.com/BelaPlatform/Bela/wiki/Building-Bela-projects-with-the-IDE}
% \end{itemize}
%\end{frame}
%
%\begin{frame}
%  \frametitle{Basic workflow}
%  \begin{itemize}
%\item Connect / disconnect computer
%\item Getting access (browser, console)
%\item Reset
%\end{itemize}
%\end{frame}
%
\usebackgroundtemplate{}
\begin{frame}
\frametitle{}
{\huge Block II\\BELA IDE }
\end{frame}
%
\begin{frame}
  \frametitle{Main functionalities}
   \begin{itemize}
\item IDE stands for ``integrated development environment'', which typically includes a source code editor, a debugger and build automation tools.
\item Bela IDE documentation: \url{https://github.com/BelaPlatform/Bela/wiki/Bela-IDE}
\end{itemize}
\end{frame}
%
%\begin{frame}
%  \frametitle{Oscilloscope}
%\end{frame}
%%
\usebackgroundtemplate{}
\begin{frame}
\frametitle{}
{\huge Block III\\Pd + BELA}
\end{frame}
%
\begin{frame}
  \frametitle{Hello World!}
       \begin{itemize}
     \item See folder ``Bela'' in code day 2.
     \end{itemize}
\end{frame}
%
%\begin{frame}
%  \frametitle{Mic / speaker through Pd: music listening}
%\end{frame}
%
\usebackgroundtemplate{}
\begin{frame}
\frametitle{}
{\huge Block III\\Fieldwork}
\end{frame}
%
\begin{frame}
  \frametitle{Fieldwork II: Embedded music listening}
        \begin{itemize}
	\item Development of a patch that applies music listening concepts.
	\item Group rehearsal so that each team member has a part or an instance of the music instrument.
    \end{itemize} 
\end{frame}
%
\usebackgroundtemplate{}
\begin{frame}
\frametitle{}
{\huge Block IV\\Rehearsal and Performance}
\end{frame}
\begin{frame}
  \frametitle{References}
  \printbibliography
\end{frame}

\end{document}
